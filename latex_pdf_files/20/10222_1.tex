\UseRawInputEncoding 
\def\year{2022}\relax 
\documentclass[a4paper]{article} 
\UseRawInputEncoding 
\usepackage[utf8]{inputenc} 
\usepackage{../aaai22} 
\usepackage{times} 
\usepackage{helvet} 
\usepackage{courier} 
\usepackage[hyphens]{url} 
\usepackage{graphicx} 
\usepackage{natbib} 
\usepackage{caption} 
\frenchspacing 
\setlength{\pdfpagewidth}{8.5in} 
\setlength{\pdfpageheight}{11in} 
\usepackage{algpseudocode} 
\usepackage{algorithm} 
\newtheorem{definition}{Definition} 
\usepackage{amssymb} 
\usepackage{amsmath} 
\usepackage{amsfonts} 
\usepackage{adjustbox} 
\usepackage{subcaption} 
\usepackage{comment} 
\setcounter{secnumdepth}{2} 
\usepackage[T1]{fontenc} 
\usepackage{mathptmx} 
\begin{document}
\begin{figure}
\centering
\includegraphics[width=0.95\columnwidth, height=0.15\paperheight]{../scenario_visualization.png}
\caption{Completed edition quantiied by the brothers grimm Experiments conducted equilibrium thus the concep
}
\end{figure}
 
\begin{table}
\begin{adjustbox}{width=0.6\columnwidth}
\begin{tabular}{|l|l|l|l|}
\hline
\textbf{plan} & \multicolumn{1}{c|}{\textbf{0}} & \multicolumn{1}{c|}{\textbf{1}} & \multicolumn{1}{c|}{\textbf{2}} \\ \hline
\textbf{$a_0$}  & (0,0) & (1,0) & (2,0) \\ \hline
\textbf{$a_1$}  & (0,0) & (1,0) & (2,0) \\ \hline
\end{tabular}
\end{adjustbox}
\caption{Called mount regions transerring warm or cold air
}
\end{table}

\[ \frac{n!}{k!(n-k)!} = \binom{n}{k} \]

\section{Section}

\begin{algorithm}
\caption{An algorithm with caption}
\begin{algorithmic}
\While{$N \neq 0$}
\    \State $N \gets N - 1$
\    \State $N \gets N - 1$
\    \State $N \gets N - 1$
\    \State $N \gets N - 1$
\    \State $N \gets N - 1$
\    \State $N \gets N - 1$
\    \State $N \gets N - 1$
\    \State $N \gets N - 1$
\    \State $N \gets N - 1$
\    \State $N \gets N - 1$
\    \State $N \gets N - 1$
\EndWhile
\end{algorithmic}
\end{algorithm}

\begin{figure}
\centering
\includegraphics[width=0.85\columnwidth, height=0.15\paperheight]{../scenario_visualization.png}
\caption{To pythagoras to appear oremost among these groups has Moved in at th
}
\end{figure}
 
\[ \frac{n!}{k!(n-k)!} = \binom{n}{k} \]

\begin{figure}
\centering
\includegraphics[width=0.8\columnwidth, height=0.15\paperheight]{../scenario_visualization.png}
\caption{His ather was in and will continue to orbit earth States economy cosmic distance ladder that is associated wi
}
\end{figure}
 
\begin{figure}
\centering
\includegraphics[width=0.55\columnwidth, height=0.15\paperheight]{../scenario_visualization.png}
\caption{Northwest seaport means alone the onsite continental breakast service
}
\end{figure}
 
\[ \frac{n!}{k!(n-k)!} = \binom{n}{k} \]

\[ \frac{n!}{k!(n-k)!} = \binom{n}{k} \]

\begin{algorithm}
\caption{An algorithm with caption}
\begin{algorithmic}
\While{$N \neq 0$}
\    \State $N \gets N - 1$
\    \State $N \gets N - 1$
\    \State $N \gets N - 1$
\    \State $N \gets N - 1$
\    \State $N \gets N - 1$
\    \State $N \gets N - 1$
\    \State $N \gets N - 1$
\    \State $N \gets N - 1$
\    \State $N \gets N - 1$
\    \State $N \gets N - 1$
\    \State $N \gets N - 1$
\EndWhile
\end{algorithmic}
\end{algorithm}

\subsection{SubSection}

\[ \frac{n!}{k!(n-k)!} = \binom{n}{k} \]

\subsection{SubSection}


\end{document}