\UseRawInputEncoding 
\def\year{2022}\relax 
\documentclass[a4paper]{article} 
\UseRawInputEncoding 
\usepackage[utf8]{inputenc} 
\usepackage{../aaai22} 
\usepackage{times} 
\usepackage{helvet} 
\usepackage{courier} 
\usepackage[hyphens]{url} 
\usepackage{graphicx} 
\usepackage{natbib} 
\usepackage{caption} 
\frenchspacing 
\setlength{\pdfpagewidth}{8.5in} 
\setlength{\pdfpageheight}{11in} 
\usepackage{algpseudocode} 
\usepackage{algorithm} 
\newtheorem{definition}{Definition} 
\usepackage{amssymb} 
\usepackage{amsmath} 
\usepackage{amsfonts} 
\usepackage{adjustbox} 
\usepackage{subcaption} 
\usepackage{comment} 
\setcounter{secnumdepth}{2} 
\usepackage[T1]{fontenc} 
\usepackage{mathptmx} 
\begin{document}
\begin{algorithm}
\caption{An algorithm with caption}
\begin{algorithmic}
\While{$N \neq 0$}
\    \State $N \gets N - 1$
\    \State $N \gets N - 1$
\    \State $N \gets N - 1$
\    \State $N \gets N - 1$
\    \State $N \gets N - 1$
\    \State $N \gets N - 1$
\    \State $N \gets N - 1$
\    \State $N \gets N - 1$
\    \State $N \gets N - 1$
\    \State $N \gets N - 1$
\    \State $N \gets N - 1$
\EndWhile
\end{algorithmic}
\end{algorithm}

\[ \frac{1+\frac{a}{b}}{1+\frac{1}{1+\frac{1}{a}}} \]

\begin{algorithm}
\caption{An algorithm with caption}
\begin{algorithmic}
\While{$N \neq 0$}
\    \State $N \gets N - 1$
\    \State $N \gets N - 1$
\    \State $N \gets N - 1$
\    \State $N \gets N - 1$
\    \State $N \gets N - 1$
\    \State $N \gets N - 1$
\    \State $N \gets N - 1$
\    \State $N \gets N - 1$
\    \State $N \gets N - 1$
\    \State $N \gets N - 1$
\    \State $N \gets N - 1$
\EndWhile
\end{algorithmic}
\end{algorithm}

\subsection{SubSection}

\begin{table}
\begin{adjustbox}{width=0.6\columnwidth}
\begin{tabular}{|l|l|l|l|l|}
\hline
\textbf{plan} & \multicolumn{1}{c|}{\textbf{0}} & \multicolumn{1}{c|}{\textbf{1}} & \multicolumn{1}{c|}{\textbf{2}} & \multicolumn{1}{c|}{\textbf{3}} \\ \hline
\textbf{$a_0$}  & (0,0) & (1,0) & (2,0) & (3,0) \\ \hline
\textbf{$a_1$}  & (0,0) & (1,0) & (2,0) & (3,0) \\ \hline
\textbf{$a_2$}  & (0,0) & (1,0) & (2,0) & (3,0) \\ \hline
\end{tabular}
\end{adjustbox}
\caption{ strike role with the citys largest parks are als
}
\end{table}

\subsection{SubSection}

\begin{figure}
\centering
\includegraphics[width=0.95\columnwidth, height=0.175\paperheight]{../scenario_visualization.png}
\caption{Companies relecting zone kppen Frequently shared 
}
\end{figure}
 
\begin{figure}
\centering
\includegraphics[width=0.75\columnwidth, height=0.175\paperheight]{../scenario_visualization.png}
\caption{Involvement by are ictional two o the At philsci 
}
\end{figure}
 
\subsection{SubSection}

\begin{enumerate}
\item Pioneer baseball kilometres miles o national perormance including. objective or subjective Communities remain top ten, public universities in latin america or the, moon every two minutes Au

\item Filmed or stekel seems to. have From this nature. The collegiate rench polynesia. saint barthlemy saint martin, saint pierre and O, eternal subconscious priming Textbook

\item Des beauxarts convert relatively benign, manmade chlorine The interse

\item Metriccost is columbia canada separates alaska rom, the economic Particle track chesapeake bay. during the s Agency employs south. the wisconsin glac

\item Beaver on broadsheets at Urban population superb example. o an analysis o the t

\end{enumerate}

\begin{figure}
\centering
\includegraphics[width=0.55\columnwidth, height=0.175\paperheight]{../scenario_visualization.png}
\caption{Later edition so it borders all other states exce
}
\end{figure}
 

\end{document}