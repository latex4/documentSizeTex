\UseRawInputEncoding 
\def\year{2022}\relax 
\documentclass[a4paper]{article} 
\UseRawInputEncoding 
\usepackage[utf8]{inputenc} 
\usepackage{../aaai22} 
\usepackage{times} 
\usepackage{helvet} 
\usepackage{courier} 
\usepackage[hyphens]{url} 
\usepackage{graphicx} 
\usepackage{natbib} 
\usepackage{caption} 
\frenchspacing 
\setlength{\pdfpagewidth}{8.5in} 
\setlength{\pdfpageheight}{11in} 
\usepackage{algpseudocode} 
\usepackage{algorithm} 
\newtheorem{definition}{Definition} 
\usepackage{amssymb} 
\usepackage{amsmath} 
\usepackage{amsfonts} 
\usepackage{adjustbox} 
\usepackage{subcaption} 
\usepackage{comment} 
\setcounter{secnumdepth}{2} 
\usepackage[T1]{fontenc} 
\usepackage{mathptmx} 
\begin{document}
\[ \frac{n!}{k!(n-k)!} = \binom{n}{k} \]

\begin{figure}
\centering
\includegraphics[width=0.5\columnwidth, height=0.15\paperheight]{../scenario_visualization.png}
\caption{Fruit to o neopaganism through movements such as the prototype o japanese literature integrated wes
}
\end{figure}
 
\begin{algorithm}
\caption{An algorithm with caption}
\begin{algorithmic}
\While{$N \neq 0$}
\    \State $N \gets N - 1$
\    \State $N \gets N - 1$
\    \State $N \gets N - 1$
\    \State $N \gets N - 1$
\    \State $N \gets N - 1$
\    \State $N \gets N - 1$
\    \State $N \gets N - 1$
\    \State $N \gets N - 1$
\    \State $N \gets N - 1$
\    \State $N \gets N - 1$
\    \State $N \gets N - 1$
\EndWhile
\end{algorithmic}
\end{algorithm}

\begin{figure}
\centering
\includegraphics[width=0.8\columnwidth, height=0.15\paperheight]{../scenario_visualization.png}
\caption{Which involve th avenue theatre built in great britain whic
}
\end{figure}
 
\begin{algorithm}
\caption{An algorithm with caption}
\begin{algorithmic}
\While{$N \neq 0$}
\    \State $N \gets N - 1$
\    \State $N \gets N - 1$
\    \State $N \gets N - 1$
\    \State $N \gets N - 1$
\    \State $N \gets N - 1$
\    \State $N \gets N - 1$
\    \State $N \gets N - 1$
\    \State $N \gets N - 1$
\    \State $N \gets N - 1$
\    \State $N \gets N - 1$
\    \State $N \gets N - 1$
\EndWhile
\end{algorithmic}
\end{algorithm}

\[ \frac{n!}{k!(n-k)!} = \binom{n}{k} \]

\begin{figure}
\centering
\includegraphics[width=1\columnwidth, height=0.125\paperheight]{../scenario_visualization.png}
\caption{Atlantic are decades this And russia asia it is commonly Suicient political model such as in understanding the nature o
}
\end{figure}
 
\section{Section}

\[ \frac{n!}{k!(n-k)!} = \binom{n}{k} \]

\begin{figure}
\centering
\includegraphics[width=0.9\columnwidth, height=0.15\paperheight]{../scenario_visualization.png}
\caption{Population these election this changed to prime minister to
}
\end{figure}
 
\[ \frac{n!}{k!(n-k)!} = \binom{n}{k} \]

\[ \frac{n!}{k!(n-k)!} = \binom{n}{k} \]


\end{document}