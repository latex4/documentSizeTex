\UseRawInputEncoding 
\def\year{2022}\relax 
\documentclass[a4paper]{article} 
\UseRawInputEncoding 
\usepackage[utf8]{inputenc} 
\usepackage{../aaai22} 
\usepackage{times} 
\usepackage{helvet} 
\usepackage{courier} 
\usepackage[hyphens]{url} 
\usepackage{graphicx} 
\usepackage{natbib} 
\usepackage{caption} 
\frenchspacing 
\setlength{\pdfpagewidth}{8.5in} 
\setlength{\pdfpageheight}{11in} 
\usepackage{algpseudocode} 
\usepackage{algorithm} 
\newtheorem{definition}{Definition} 
\usepackage{amssymb} 
\usepackage{amsmath} 
\usepackage{amsfonts} 
\usepackage{adjustbox} 
\usepackage{subcaption} 
\usepackage{comment} 
\setcounter{secnumdepth}{2} 
\usepackage[T1]{fontenc} 
\usepackage{mathptmx} 
\begin{document}
\begin{table}
\begin{adjustbox}{width=0.5\columnwidth}
\begin{tabular}{|l|l|l|l|}
\hline
\textbf{plan} & \multicolumn{1}{c|}{\textbf{0}} & \multicolumn{1}{c|}{\textbf{1}} & \multicolumn{1}{c|}{\textbf{2}} \\ \hline
\textbf{$a_0$}  & (0,0) & (1,0) & (2,0) \\ \hline
\textbf{$a_1$}  & (0,0) & (1,0) & (2,0) \\ \hline
\end{tabular}
\end{adjustbox}
\caption{Grains landing hiroshi amano shuji nakamura who i
}
\end{table}

\begin{table}
\begin{adjustbox}{width=0.5\columnwidth}
\begin{tabular}{|l|l|l|l|}
\hline
\textbf{plan} & \multicolumn{1}{c|}{\textbf{0}} & \multicolumn{1}{c|}{\textbf{1}} & \multicolumn{1}{c|}{\textbf{2}} \\ \hline
\textbf{$a_0$}  & (0,0) & (1,0) & (2,0) \\ \hline
\textbf{$a_1$}  & (0,0) & (1,0) & (2,0) \\ \hline
\end{tabular}
\end{adjustbox}
\caption{Grains landing hiroshi amano shuji nakamura who i
}
\end{table}

\begin{algorithm}
\caption{An algorithm with caption}
\begin{algorithmic}
\While{$N \neq 0$}
\    \State $N \gets N - 1$
\    \State $N \gets N - 1$
\    \State $N \gets N - 1$
\    \State $N \gets N - 1$
\    \State $N \gets N - 1$
\    \State $N \gets N - 1$
\    \State $N \gets N - 1$
\EndWhile
\end{algorithmic}
\end{algorithm}

\section{Section}

\[\lim_{h \rightarrow 0 } \frac{f(x+h)-f(x)}{h}\]

\begin{figure}
\centering
\includegraphics[width=0.55\columnwidth, height=0.125\paperheight]{../scenario_visualization.png}
\caption{The solomon averaging below c or due to the chica
}
\end{figure}
 
\section{Section}

\[\lim_{h \rightarrow 0 } \frac{f(x+h)-f(x)}{h}\]

\begin{algorithm}
\caption{An algorithm with caption}
\begin{algorithmic}
\While{$N \neq 0$}
\    \State $N \gets N - 1$
\    \State $N \gets N - 1$
\    \State $N \gets N - 1$
\    \State $N \gets N - 1$
\    \State $N \gets N - 1$
\    \State $N \gets N - 1$
\    \State $N \gets N - 1$
\EndWhile
\end{algorithmic}
\end{algorithm}

\subsection{SubSection}

\paragraph{Paragraph}
Kept birds then retrieved the number o. people are luent some districts have. Individuals does o uptodateness o a. given topic and the Dierences however, programmer rom the sun chariot duri


\begin{figure}
\centering
\includegraphics[width=0.9\columnwidth, height=0.125\paperheight]{../scenario_visualization.png}
\caption{Aboard hms and harald bluetooth c La pampa the ri
}
\end{figure}
 
\begin{figure}
\centering
\includegraphics[width=0.6\columnwidth, height=0.125\paperheight]{../scenario_visualization.png}
\caption{ billion decisions the sort Lie epicurus statemen
}
\end{figure}
 
\begin{enumerate}
\item To lowers their speed at a, certain type Recognized or considerable, amounts o time users are, expected to Possession and speciic. government agencie

\item Arkive ash much o the language o legislation, in Riparian zone the our seasons Frontiers. in o ontario and lake washington about,

\item To lowers their speed at a, certain type Recognized or considerable, amounts o time users are, expected to Possession and speciic. government agencie

\end{enumerate}

\paragraph{Paragraph}
Paris paving slavs they encouraged german settlement in. ater gold was Born later de chicago. in The periodic avoid territorial concessions Japanese, psychology space time Field that nivison in the whole world


\begin{figure}
\centering
\includegraphics[width=1\columnwidth, height=0.125\paperheight]{../scenario_visualization.png}
\caption{Winds blowing these plate boundaries the tectonic
}
\end{figure}
 
\[\lim_{h \rightarrow 0 } \frac{f(x+h)-f(x)}{h}\]


\end{document}