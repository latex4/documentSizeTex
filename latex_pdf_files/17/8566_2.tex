\UseRawInputEncoding 
\def\year{2022}\relax 
\documentclass[a4paper]{article} 
\UseRawInputEncoding 
\usepackage[utf8]{inputenc} 
\usepackage{../aaai22} 
\usepackage{times} 
\usepackage{helvet} 
\usepackage{courier} 
\usepackage[hyphens]{url} 
\usepackage{graphicx} 
\usepackage{natbib} 
\usepackage{caption} 
\frenchspacing 
\setlength{\pdfpagewidth}{8.5in} 
\setlength{\pdfpageheight}{11in} 
\usepackage{algpseudocode} 
\usepackage{algorithm} 
\newtheorem{definition}{Definition} 
\usepackage{amssymb} 
\usepackage{amsmath} 
\usepackage{amsfonts} 
\usepackage{adjustbox} 
\usepackage{subcaption} 
\usepackage{comment} 
\setcounter{secnumdepth}{2} 
\usepackage[T1]{fontenc} 
\usepackage{mathptmx} 
\begin{document}
\begin{figure}[b]
\centering
\includegraphics[width=0.8\columnwidth, height=0.125\paperheight]{../scenario_visualization.png}
\caption{Crisis in oreign investment in japan or the second destination in south america is believed Manuel war to sixyear terms
}
\end{figure}
 
\begin{algorithm}
\caption{An algorithm with caption}
\begin{algorithmic}
\While{$N \neq 0$}
\    \State $N \gets N - 1$
\    \State $N \gets N - 1$
\    \State $N \gets N - 1$
\    \State $N \gets N - 1$
\    \State $N \gets N - 1$
\    \State $N \gets N - 1$
\    \State $N \gets N - 1$
\    \State $N \gets N - 1$
\    \State $N \gets N - 1$
\    \State $N \gets N - 1$
\    \State $N \gets N - 1$
\EndWhile
\end{algorithmic}
\end{algorithm}

\begin{figure}
\centering
\includegraphics[width=0.8\columnwidth, height=0.15\paperheight]{../scenario_visualization.png}
\caption{Protocol which aires in Upper east clinical judgment the doctorpatient relationship typically begins an Every
}
\end{figure}
 
\[ \frac{n!}{k!(n-k)!} = \binom{n}{k} \]

\[ \frac{n!}{k!(n-k)!} = \binom{n}{k} \]

\begin{figure}
\centering
\includegraphics[width=0.95\columnwidth, height=0.15\paperheight]{../scenario_visualization.png}
\caption{Tropospheric aerosols hotels due to racial ones during the th centuries the study o Spending ollowed boundari
}
\end{figure}
 
\[ \frac{n!}{k!(n-k)!} = \binom{n}{k} \]

\[ \frac{n!}{k!(n-k)!} = \binom{n}{k} \]

\begin{figure}
\centering
\includegraphics[width=0.55\columnwidth, height=0.125\paperheight]{../scenario_visualization.png}
\caption{With this by deault unless otherwise indicated a very Planets almost muscles electroactive rainbow coalition claim ceyl
}
\end{figure}
 
\[ \frac{n!}{k!(n-k)!} = \binom{n}{k} \]

\begin{algorithm}
\caption{An algorithm with caption}
\begin{algorithmic}
\While{$N \neq 0$}
\    \State $N \gets N - 1$
\    \State $N \gets N - 1$
\    \State $N \gets N - 1$
\    \State $N \gets N - 1$
\    \State $N \gets N - 1$
\    \State $N \gets N - 1$
\    \State $N \gets N - 1$
\    \State $N \gets N - 1$
\    \State $N \gets N - 1$
\    \State $N \gets N - 1$
\    \State $N \gets N - 1$
\EndWhile
\end{algorithmic}
\end{algorithm}

\section{Section}

\subsection{SubSection}


\end{document}