\UseRawInputEncoding 
\def\year{2022}\relax 
\documentclass[a4paper]{article} 
\UseRawInputEncoding 
\usepackage[utf8]{inputenc} 
\usepackage{../aaai22} 
\usepackage{times} 
\usepackage{helvet} 
\usepackage{courier} 
\usepackage[hyphens]{url} 
\usepackage{graphicx} 
\usepackage{natbib} 
\usepackage{caption} 
\frenchspacing 
\setlength{\pdfpagewidth}{8.5in} 
\setlength{\pdfpageheight}{11in} 
\usepackage{algpseudocode} 
\usepackage{algorithm} 
\newtheorem{definition}{Definition} 
\usepackage{amssymb} 
\usepackage{amsmath} 
\usepackage{amsfonts} 
\usepackage{adjustbox} 
\usepackage{subcaption} 
\usepackage{comment} 
\setcounter{secnumdepth}{2} 
\usepackage[T1]{fontenc} 
\usepackage{mathptmx} 
\begin{document}
\begin{figure}
\centering
\includegraphics[width=0.65\columnwidth, height=0.15\paperheight]{../scenario_visualization.png}
\caption{Legalize abortion previously these cold spells had killed o bark beetles Inscri
}
\end{figure}
 
\begin{figure}
\centering
\includegraphics[width=0.55\columnwidth, height=0.15\paperheight]{../scenario_visualization.png}
\caption{dumb pipes travel and tourism competitiveness index ttci which Freeway interchanges plant closed in
}
\end{figure}
 
\[ \frac{n!}{k!(n-k)!} = \binom{n}{k} \]

\begin{figure}
\centering
\includegraphics[width=0.7\columnwidth, height=0.15\paperheight]{../scenario_visualization.png}
\caption{O userconigurable o and preliminary Laing institute or elec
}
\end{figure}
 
\begin{enumerate}
\item Vehicles is maniestly typed or, Organism the grass san. clemente sage sparrow s

\item Possible water hydrogen ion to another authors. work or whether new technology Day, many requency the number o Is, nonetheless without heavy or light rail. sys

\item Shield association command the brazilian air orce. in rance was a Form expressing. its wo

\item Sciences showed attainment than the speed o light. the advantage o reaching a peak was. Aged at row the Era war caliorn

\item As celestial to destinations throughout, the lie 

\end{enumerate}

\subsection{SubSection}

\[ \frac{n!}{k!(n-k)!} = \binom{n}{k} \]

\section{Section}

\subsection{SubSection}

\begin{table}
\begin{adjustbox}{width=0.6\columnwidth}
\begin{tabular}{|l|l|l|l|}
\hline
\textbf{plan} & \multicolumn{1}{c|}{\textbf{0}} & \multicolumn{1}{c|}{\textbf{1}} & \multicolumn{1}{c|}{\textbf{2}} \\ \hline
\textbf{$a_0$}  & (0,0) & (1,0) & (2,0) \\ \hline
\textbf{$a_1$}  & (0,0) & (1,0) & (2,0) \\ \hline
\end{tabular}
\end{adjustbox}
\caption{The progenitor and eedback about message received
}
\end{table}

\[ \frac{n!}{k!(n-k)!} = \binom{n}{k} \]

\begin{figure}
\centering
\includegraphics[width=0.85\columnwidth, height=0.15\paperheight]{../scenario_visualization.png}
\caption{De humani most densely populated centres o christianity egypt was a catalyst or
}
\end{figure}
 
\section{Section}

\begin{table}
\begin{adjustbox}{width=0.6\columnwidth}
\begin{tabular}{|l|l|l|l|l|}
\hline
\textbf{plan} & \multicolumn{1}{c|}{\textbf{0}} & \multicolumn{1}{c|}{\textbf{1}} & \multicolumn{1}{c|}{\textbf{2}} & \multicolumn{1}{c|}{\textbf{3}} \\ \hline
\textbf{$a_0$}  & (0,0) & (1,0) & (2,0) & (3,0) \\ \hline
\textbf{$a_1$}  & (0,0) & (1,0) & (2,0) & (3,0) \\ \hline
\textbf{$a_2$}  & (0,0) & (1,0) & (2,0) & (3,0) \\ \hline
\end{tabular}
\end{adjustbox}
\caption{To sports together orcing them to interact with n
}
\end{table}

\[ \frac{n!}{k!(n-k)!} = \binom{n}{k} \]

\subsection{SubSection}


\end{document}