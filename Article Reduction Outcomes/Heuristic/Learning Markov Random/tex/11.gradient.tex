\section{Constrained MRF Model}\label{appendix:constrain-mrf}

\subsection{Single Variable Form of Constrained MRF}  \label{appendix:svf}
Here we provide an example of transforming MRF with pairwise and single potential functions into a single potential form by introducing extra variables. Given random variables $X_1,X_2,X_3$, we have the following example MRF model:
\begin{align*}
\phi_{\theta}(x_1,x_2,x_3)&=\theta_1 x_1+\theta_2 x_2+\theta_3 x_1 x_2\\
P_{\theta}(x)&=\frac{\exp(\phi_{\theta}(x_1,x_2,x_3))}{Z({\theta})}
\end{align*}

In the above formula, we have a cross term $x_1 x_2$. Two Boolean variables can have 4 different assignments in total. Therefore we can construct 4 extra Boolean variables to encode all these assignments. To illustrate, we introduce extra random variables $\hat{X}_{00}$, $\hat{X}_{01}$, $\hat{X}_{10}$, $\hat{X}_{11}$. We further introduce extra constraints:
 When $X_1=0,X_2=0$, the extra variable must take values: $\hat{X}_{00}=1$, $\hat{X}_{01}=0$, $\hat{X}_{10}=0$, $\hat{X}_{11}=0$. See the rest constraints in Table~\ref{tab:pairwise-to-single}.

\begin{table}[!ht]
    \centering
    \begin{tabular}{cc|cccc}
    \hline
        $X_1$ & $X_2$ &$\hat{X}_{00}$, $\hat{X}_{01}$, $\hat{X}_{10}$, $\hat{X}_{11}$  \\ \hline
        $0$ & $0$ & $1, 0, 0, 0$\\ 
        $0$ & $1$ & $0, 1, 0, 0$\\ 
        $1$ & $0$ & $0, 0, 1, 0$\\ 
        $1$ & $1$ & $0, 0, 0, 1$\\ 
    \hline
    \end{tabular}
    \caption{4 constraints for converting pairwise terms in the potential function into single variable form.}
    \label{tab:pairwise-to-single}
\end{table}
{Then the new potential function, including extended variables and pairwise to single variable constraints $\mathcal{C}$, is reformulated as:}
\begin{align*}
\hat{\phi}_{\theta}(x_1,x_2,x_3,\hat{x}_{00},\hat{x}_{01},\hat{x}_{10},\hat{x}_{11})&=\theta_1 x_1+\theta_2 x_2+\theta_3 \hat{x}_{00}+ \theta_3 \hat{x}_{01}+\theta_3 \hat{x}_{10}+\theta_3 \hat{x}_{11}\\
P_{\theta}(x|\mathcal{C})&=\frac{\exp(\hat{\phi}_{\theta}(x_1,x_2,x_3,\hat{x}_{00},\hat{x}_{01},\hat{x}_{10},\hat{x}_{11}))}{Z_{\mathcal{C}}(\theta)}
\end{align*}
For clarity, the newly added constraints do not impact Condition~\ref{cond:extreme}. Since the single variable transformation in the MRFs model originates from~\citet{Sang2005}, thus is not considered as our contribution.



\subsection{Gradient of $\log$-Partition Function $\nabla\log Z_\mathcal{C}(\theta)$}\label{appendix:partition}
We use the Chain rule of the gradient to give a detailed deduction of Equation~\eqref{eq:gradient}.
\begin{equation}\label{eq:full-gradient}
\begin{aligned}
\nabla\log Z_\mathcal{C}(\theta)=\frac{\nabla Z_\mathcal{C}(\theta)}{Z_\mathcal{C}(\theta)}=\frac{1}{Z_\mathcal{C}(\theta)}\nabla\sum_{x\in \mathcal{X}}\exp\left( \phi_{\theta}(x)\right)C(x)=\sum_{x\in\mathcal{X}} \frac{\exp(\phi_{\theta}(x))C(x) }{Z_\mathcal{C}(\theta)}{\nabla\phi_\theta(x)} =&\sum_{x\in \mathcal{X}} P_\theta(x|\mathcal{C}) \nabla\phi_{\theta}(x) \\
=&\mathbb{E}_{x\sim P_{\theta}(\tilde{x}|\mathcal{C})} \left({\nabla\phi_{\theta}(x)}\right)
\end{aligned}
\end{equation}
{The above result shows the gradient of the constrained partition function is equivalent to the expectation of the gradient of the potential function $\nabla\phi_{\theta}$ over the model's distribution (\textit{i.e.}, $P_{\theta}(\tilde{x}|\mathcal{C})$). Therefore, we transform the gradient estimation problem into the problem of sampling from the current MRF model.}


